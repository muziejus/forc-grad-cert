% Based off https://github.com/kjhealy/pandoc-templates/blob/master/templates/xelatex.template
\documentclass[%
  ,
  article,
  ,
  oneside
  ]{memoir}

%%% Font selection
\usepackage{fontspec} % Choose fonts intelligently
%%% Fontspec looks for the font set in metadata.yml
%%%
%%% Included below are some advanced OpenType features that work well with
%%% EB Garamond. You can download EB Garamond here:
%%%
%%% http://www.georgduffner.at/ebgaramond/download.html
%%%
%%% For more information on the OpenType features available for declaration,
%%% please see part IV of the Fontspec documentation, available here:
%%%
%%% http://mirrors.ctan.org/macros/latex/contrib/fontspec/fontspec.pdf
\setmainfont{EB Garamond}[%
  %%% These features may not work with all fonts.
        Mapping=tex-text
]

%%% PDF generation
\usepackage[ breaklinks=true, hidelinks ]{hyperref}
\hypersetup{%
            pdftitle={Proposal for a Graduate Certificate in Research
Computing},
            pdfauthor={Dennis Y. Tenen, Roger Creel},
            pdfborder={0 0 0},
            breaklinks=true}

%%% Layout of the page
  \setlrmarginsandblock{1in}{1in}{*}
  \setulmarginsandblock{1in}{1in}{*}
  \setheadfoot{\baselineskip}{\baselineskip} % footer is a baseline tall.
  \setheaderspaces{*}{6pt}{*}

\checkandfixthelayout

%%% Title layout
\pretitle{\begin{flushleft}\LARGE\textsc}
\posttitle{\par\end{flushleft}\vskip 0.5em}
\preauthor{\vskip 1em \begin{flushleft}\begin{tabular}[t]{l}}
\renewcommand*\and{\\}
\postauthor{\end{tabular}\end{flushleft}}
\predate{}
\postdate{}
\date{}
% \preauthor{\vskip 1em }
% \postauthor{}
% \renewcommand*\and{\newline \parindent}
% \preauthor{\begin{flushleft}%
%            \large \lineskip 0.5em%
%            \begin{tabular}[t]{@{}l}}
% \postauthor{\end{tabular}\end{flushleft}}
% \renewcommand*\and{%
%   \end{tabular}%
%   \vskip 1em \relax
%   \begin{tabular}[t]{l}}
% \renewcommand*{\andnext}{%
%   \end{tabular}\\ \begin{tabular}[t]{@{}l}}
% \preauthor{\begin{flushleft} 0.5em}
% \postauthor{\par\end{flushleft}}
\setlength{\droptitle}{-48pt}

\title{Proposal for a Graduate Certificate in Research Computing}
\author{ Dennis Y. Tenen, \emph{Associate Professor of English and
Comparative Literature, Columbia University}
 \and  Roger Creel, \emph{PhD Candidate, Department of Earth and
Environmental Sciences, Columbia University}
}

%%% Graphics
\usepackage{tikz}

% \list{--}{\topsep=2pt\itemsep=0pt\parsep=0pt\parskip=0pt\labelwidth=8pt\leftmargin=8pt\itemindent=0pt\labelsep=2pt}
% \endlist

\def\coursetrack#1#2#3#4{
  \node (#2) [branch] {
  \Large #1 Track\\
  };
}


%%% Sections
% This removes numbering from sections by redefining \section to mimic \section*
\let\oldsection\section
\renewcommand{\section}[1]{\oldsection*{#1}}

%%% Other formatting
\usepackage{enumitem}

\begin{document}

\maketitle



The growth of interdisciplinary, cutting-edge research that relies on
research computing (RC), coupled with the increasingly computationally
driven job market for doctoral students, presents an opportunity for
Columbia University to produce PHDs who are ready and able to meet these
needs, regardless of their career track. In offering a graduate
certificate in research computing to doctoral students across the
university, Columbia will aid their students as they prepare for
distinguished careers. Additionally, the certification process invites
participating PhD students to join the interdisciplinary community of
scholars at Columbia, enabling new opportunities for distinctive
research and collaboration.

Building on the popularity of the Foundations for Research Computing
two-day sessions, we propose a certification process where all
applicants will take an introductory course on RC with Python before
branching off to continue their coursework with approved courses closer
to their fields of study.

\def\textualtrack{

}

\usetikzlibrary {positioning}
\begin{tikzpicture}[
    level distance=48pt,
    sibling distance=2.25in,
    edge from parent path=
    {(\tikzparentnode.south) .. controls +(0, -1) and +(0,1)
                             .. (\tikzchildnode.north)},
    intro/.style={
      anchor=south,
      inner sep=8pt,
      rectangle,
      minimum width=6in,
      rounded corners=3mm,
      very thick,
      draw=black,
    },
    branch/.style={
      anchor=north,
      align=left,
      inner sep=8pt,
      rectangle,
      width=2in,
      rounded corners=3mm,
      very thick,
      draw=black!50,
    }
]
  \node (intro) [intro] {
    \begin{tabular}{c}
      \LARGE \textsc{Introduction to Research Computing}\\
      \normalsize course description \textit{course description} course description
    \end{tabular}
  }
    child {node (textual) [branch] {\Large \textsc{Textual track}\\
Builds on current offerings in textual data analysis and natural language processing
    }}   
    child {node (numerical) [branch] {\Large \textsc{Numerical track}\\Course 1}}   
    child {node (datasci) [branch] {\Large \textsc{Data Science track}\\Course 1}
    };   
\end{tikzpicture}

\hypertarget{summary}{%
\subsection{Summary}\label{summary}}

Considering the growing prevalence of interdisciplinary, cutting-edge
research that relies on research computing (RC) as well as an
increasingly computationally driven job market outside of the
professoriat, Columbia University must produce PhDs who are ready and
able to meet these needs, regardless of their career track. In offering
a graduate certificate in research computing, Columbia will continue to
meet the needs of world-class PhD students as they distinguish
themselves with their research. Additionally, the certification process
invites participating PhD students to join the interdisciplinarily
minded community of scholars at Columbia, enabling new avenues for
distinctive research and collaboration.

With this document, we propose that the Graduate School of Arts and
Sciences, with Office of the Executive Vice President for Research,
Columbia University Information Technologies, and the Columbia
University Libraries, establish a formal graduate certificate in
research computing for PhD students.

Our suggestions are preliminary and provide only a launchpad for the
work of a more formal committee featuring the members of the bodies
mentioned above. We imagine, however, a set of learning objectives
similar to what follows:

\begin{itemize}
\tightlist
\item
  Describe and extend the qualities of exceptional research computing
\item
  Analyze, critique, and question inherent assumptions characteristic of
  research computing
\item
  Interpret, assess, and argue for strategies for managing research data
\item
  Synthesize, defend, and support an ethically driven, critical
  computation
\item
  Design, develop, and generate a computationally intensive doctoral
  research project
\item
  Produce a presentation that illustrates and models exceptional
  research computing
\end{itemize}

Students will meet the objectives above through four modules:

\begin{itemize}
\tightlist
\item
  A stand-alone, required, for-credit course introducing research
  computing in theory and practice
\item
  Supplemental, elective courses with a research computing focus drawn
  from already existing offerings
\item
  Participation in non-credit workshops and training provided by the
  Libraries and/or CUIT
\item
  Presenting, in public, aspects of their computationally intensive
  doctoral research project
\end{itemize}

\hypertarget{need}{%
\subsection{Need}\label{need}}

During the hiring year of 2021--2022, we identified 29 assistant
professor openings across the world in earth and environmental science
(climate science, geophysics, geoinformatics, etc.). Of these, 26
listings (90\%) strongly prefer candidates with a research computing
background and 9 (31\%) require that expertise. Similarly, at Columbia,
research computing fluency is a de facto requirement for graduation in
earth and environmental science, but curricular opportunities for
acquiring skills in RC are limited and ad hoc.

In a related vein, during the hiring year of 2020--2021, we counted 28
positions (15 tenure-track) across the world that were hiring for
someone in the humanities with research computational skills, such as
Assistant Professors of Digital Humanities, Assistant Professors in
Social Justice Informatics, or non-faculty positions supporting
computing and digital scholarship in the humanities, such as Digital
Humanities Project Officers or Digital Scholarship Coordinators. Even
though fluency in RC is not expected among graduate students in the
humanities at Columbia, these findings indicate that the University
could support its students better in equipping them to apply for these
positions.

\hypertarget{provisional-curriculum}{%
\subsection{Provisional Curriculum}\label{provisional-curriculum}}

As stated above, the curriculum would be designed and overseen by a
faculty-led advisory board. In the current absence of such a board, we
propose a provisional curriculum, where graduate students earn the
certificate through a mix of pedagogical moments, including courses for
credit, non-credit workshops, and public scholarship.

First, at least one for-credit course would be a required course for all
students seeking certification. This course, tentatively named
``Introduction to Research Computing,'' would establish the social and
environmental contexts for research computing, especially surrounding
research design and data management. Upon completing this course,
students will be able to consider the effects of various research
computing projects, including their own, with a critical eye toward
greater accountability. This course would also incorporate the short,
web-based course ``The Whys and Hows of Exceptional Scholarship with
Research Data,'' prepared by the Libraries and the Columbia Center for
Teaching and Learning, and funded by a Provost Interdisciplinary
Teaching Award for 2021--2023.

A second required course, ``Research Computing in Public,'' could
reposition the contextual concerns towards doing research computing in
the public sphere, in the sense of analyzing and presenting research
findings with the public as an intended audience or stakeholder.

Additionally, the certification process would include a public-facing
project tied to the students' own research. We envision a public event
showcasing the student's work with presentations and guidance on hosting
their work online.

We also foresee students' taking a number of elective courses. These
courses would come from an annually generated list of courses. They
would not be offered explicitly for students pursuing the GCRC, but the
faculty would know to expect such students in their classes. The
advisory board would oversee the collection of appropriate electives.

Finally, students would enrich their knowledge of RC through various
workshops offered on campus, including those offered by Foundations for
Research Computing and the Libraries. The content of these workshops can
also be integrated into the required, for-credit courses. The workshops,
however, will not be considered as sufficient instruction in how to do
computational research. That is, they are supplemental to programming
instruction in the for-credit curriculum and not a replacement for
programming instruction.

\hypertarget{audience}{%
\subsection{Audience}\label{audience}}

We imagine that the initial population of interested students will be
GSAS PhD students still completing their coursework, with some
exceptions who are already at the proposal stage or later, enrolling
with appropriate permission from their adviser(s). There is no
prerequisite level of computational knowledge required. Students with no
programming experience should be able to earn certification by
completing the program.

We seek students who are engaged in or planning a computationally driven
research project as well as students eager to learn about research
computing for future projects.

We also seek students who are motivated to consider the social and
ethical ramifications of research computing, illustrated in their
participation in the public-facing final project.

\hypertarget{supplemental-materials}{%
\subsection{Supplemental Materials}\label{supplemental-materials}}

\hypertarget{potential-advisory-board}{%
\subsubsection{Potential Advisory
Board}\label{potential-advisory-board}}

\begin{itemize}
\tightlist
\item
  \href{http://github.com/rabernat}{Ryan Abernathey}
\item
  Matt Jones (History)
\item
  Dennis Tenen (English/Comp Lit)
\item
  \href{https://barnard.edu/profiles/rebecca-wright}{Rebecca Wright}
\item
  Saima Akhtar (Assoc. Director of Vagelos Computational Science Center)
\item
  Chris Marianetti
\item
  Adrian Hill, Executive Director for Research Planning and Development
\item
  Reshmi Mukherjee, Vice Provost, Academic Research and Centers, Barnard
\item
  CTL (Caitlin DeClercq)
\item
  Teachers College Digital Futures Institute:

  \begin{itemize}
  \tightlist
  \item
    Lalitha Vasudevan, Managing Director and Vice Dean for Digital
    Inovation
  \item
    Rochelle Thomas, Directory of Strategy, Planning and Operation
  \item
    Abdul Malik Muftau, Project Operations Lead
  \end{itemize}
\item
  \href{https://www.marksantolucito.com/teaching.html}{Mark Santolucito
  (Barnard)}
\end{itemize}

\hypertarget{current-state-of-introductory-instruction-in-research-computing-at-columbia}{%
\subsubsection{Current state of introductory instruction in research
computing at
Columbia}\label{current-state-of-introductory-instruction-in-research-computing-at-columbia}}

\begin{itemize}
\tightlist
\item
  Foundations' Children:

  \begin{itemize}
  \tightlist
  \item
    Mechanical Engineering reached out to FRC to create bootcamps to run
    for incoming masters students. Patrick worked with Arvind\ldots, who
    ran the workshops, etc. FRC provided a logistic support and trained
    Arvind in Software Carpentries. Patrick also found teachers who
    would be paid by MechEe. This has been run a few times and is
    self-sustaining.
  \item
    LEAP: \href{https://earth.columbia.edu/projects/view/2327}{Learning
    Earth with Artificial Intelligence and Physics}. Pierre Gentine,
    lead PI. Machine learning and climate science Tian Zheng received a
    grant to\ldots. biggest grant awarded to Morningside Campus. Idea is
    to run bootcamps for a currently unknown (to us) audience.
  \end{itemize}
\item
  Other stuff:

  \begin{itemize}
  \tightlist
  \item
    Computing in Context, etc.
  \item
    Women in STEM in SIPA
  \item
    Women in Science at Columbia. (currently changing around. Who
    knows.)
  \end{itemize}
\item
  For-credit courses offered:

  \begin{itemize}
  \tightlist
  \item
    Research Computing in Earth Science
    (\href{https://rabernat.github.io/research_computing/}{Old link})
  \item
    \href{https://www.coursicle.com/columbia/courses/BIST/P6110/}{Statistical
    Computing with SAS} (Public Health course)
  \item
    \href{https://www8.gsb.columbia.edu/courses/phd/2018/fall/b9122-001}{Computing
    for Business Research} (PhD level)
  \item
    \href{https://www.coursicle.com/columbia/courses/PHYS/G6080/}{Scientific
    Computing} Mawhinney
  \item
    \href{https://www.coursicle.com/columbia/courses/ENGI/E1006/}{Introduction
    to Computing for Engineers and Applied Scientists}
  \item
    {[}Probability \& Statistics for Data Science{]}
    (http://www.math.columbia.edu/\textasciitilde fts/2021\%20W5701\%20Probability\%20and\%20Statistics\%20for\%20Data\%20Science\%20Syllabus\%20Fall\%202021\%20v1b.pdf)(Tat
    Sang Fung)
  \item
    \href{https://www.coursicle.com/columbia/courses/ORCA/E2500/}{Foundations
    of Data Science}(Yi Zhang, Xuan Zhang)
  \end{itemize}
\end{itemize}

\hypertarget{list-of-non-intro-level-research-computing-focused-courses-in-20212022}{%
\subsubsection{List of non-intro-level research computing focused
courses in
2021--2022}\label{list-of-non-intro-level-research-computing-focused-courses-in-20212022}}

{[}distinguish by dept or something{]}

\begin{itemize}
\tightlist
\item
  New Directions in Computing (Undergrad Courses in CS/Barnard)

  \begin{itemize}
  \tightlist
  \item
    \href{https://www.marksantolucito.com/COMS3997/fall2021/}{Computing
    in the Arts}
  \item
    \href{https://www.marksantolucito.com/COMS3930/syllabus.pdf}{Creative
    Embedded Systems}
  \end{itemize}
\item
  \href{http://www.columbia.edu/cu/bulletin/uwb/subj/APMA/E4300-20211-001/}{Introduction
  to Numerical Methods} (Spiegelman) (prereqs: calc, differential
  equations, linear algebra, introduction to computing for engineers)
\item
  \href{https://www.coursicle.com/barnard/courses/FYSB/BC1736/}{Tech
  Society: Good, Bad, \& Other} (Wright)
\item
  \href{https://www.coursicle.com/barnard/courses/COMS/BC3420/}{Privacy
  in a networked world} (Wright)
\item
  \href{https://www.coursicle.com/columbia/courses/CSOR/W4246/}{Algorithms
  in Data Science} (prereqs: programming, calc, diff eq)
\item
  \href{https://www.coursicle.com/barnard/courses/COMS/BC3420/}{Object
  Oriented Programming and Design in Java}
\item
  \href{https://www.coursicle.com/columbia/courses/IEOR/E4501/}{Tools
  for Analytics} (Industrial Engineering) (Julian Berman)
\item
  \href{http://journalism.columbia.edu/data}{Data Journalism}
\item
  \href{http://compjournalism.com/?p=206}{Computational Journalism}
\item
  {[}Time Series, Panel Data, \& Forecasting {]}
  (https://www.coursicle.com/columbia/courses/QMSS/G5016/) (Gregory
  Eirich)
\item
  {[}Statistical Inference \& Modeling{]}
  (https://www.coursicle.com/columbia/courses/STAT/G5703/) (prereq: prob
  \& stats for data science) (Yunxiau, Marco Avella)
\item
  \href{https://www.coursicle.com/barnard/courses/STAT/GU4205/}{Linear
  Regression Models}
\item
  \href{https://www.coursicle.com/barnard/courses/STAT/GU4221/}{Time
  Series Analysis}
\item
  \href{http://www.stat.columbia.edu/~madigan/G6102/}{Statistical
  Modeling for Data Analysis}
\item
  \href{https://www.coursicle.com/columbia/courses/STAT/W4243/}{Applied
  Data Science}(Ying Liu, Tian Zheng, David Shilane)
\item
  \href{http://www.columbia.edu/cu/bulletin/uwb/subj/APMA/E4302-20213-001/}{Methods
  in Computational
  Science}\href{https://github.com/mandli/methods-in-computational-science}{github}(Kyle
  Mandlii)
\item
  \href{https://www.marksantolucito.com/COMS3430/fall2021/}{Computational
  Sound}(Mark Santalucito)
\item
  \href{http://www.cs.columbia.edu/~cs4252/}{Intro-Computational
  Learning Theory}(Rocco Servedio)
\item
  \href{http://www.columbia.edu/cu/bulletin/uwb-test/subj/MECS/E4510-20213-001/}{Evolutionary
  Computation and Design}(Hodd Lipson)
\item
  \href{https://tonydear.github.io/teaching/coms3251}{Computational
  Linear Algebra}(Tony Dear)
\item
  \href{https://www.coursicle.com/columbia/courses/EESC/W3400/}{Computational
  Earth Science}(Kerry Key)
\item
  \href{https://www.coursicle.com/columbia/courses/IEOR/E4799/}{MSFE
  Quantitative and Computational Bootcamp}(Michael Miller, Sebastien
  Donadio)
\item
  \href{https://www.coursicle.com/columbia/courses/STAT/W4224/}{Bayesian
  Statistics}(prereq STAT GU4204 or class on theory of statistical
  inference)(Ronald Neath)
\end{itemize}

\hypertarget{analysis-of-the-2022-job-market-for-tenure-track-positions-in-earth-sciences-and-digital-humanities}{%
\subsubsection{Analysis of the 2022 job market for tenure-track
positions in Earth Sciences and Digital
Humanities}\label{analysis-of-the-2022-job-market-for-tenure-track-positions-in-earth-sciences-and-digital-humanities}}

(include CSVs)

\hypertarget{comparable-programs-at-peer-institutions}{%
\subsubsection{Comparable programs at peer
institutions}\label{comparable-programs-at-peer-institutions}}

\hypertarget{etc.}{%
\subsection{Etc.}\label{etc.}}

the opportunity for a university-wide certificate in RC

(why)

who - advisers and workers

what

for whom

\begin{center}\rule{0.5\linewidth}{0.5pt}\end{center}

Notes:

Problem w/ summer institute is figuring out how to do admin.

how do you pay administrators?

CTL, SPS, and GSAPP have summer options

state-level grad certificate program through GSAS

this would be workshops but also courses

Mellon grant for summer institute? 3 years of startup cash?

Certificate in Research Computing:

\begin{itemize}
\tightlist
\item
  Advisory board
\item
  recurring courses that would satisfy the cert
\item
  contact depts and ask if we can send students to their courses
\item
  contact chairs
\item
  certification:

  \begin{itemize}
  \tightlist
  \item
    requires a curriculum
  \item
    ``if we did this this year,'' how would it look?
  \item
    a picture that convinces CU and NY
  \item
    recurring accreditation admin lifting
  \item
    needs support letters
  \item
    begets a summer institute
  \end{itemize}
\item
  Pull job lists for stuff that would benefit from this cert (academic
  plus, etc., data skills needed for more and more tt positions)
\item
  how is this not QMSS?

  \begin{itemize}
  \tightlist
  \item
    QMSS has alternate tracks, a few reqs, then electives in AMSS
  \end{itemize}
\end{itemize}

What money is needed for a certificate?

Write for Ann, Jonathan, and Carlos Executive Summary

\end{document}
